\documentclass[conference]{IEEEtran}
\IEEEoverridecommandlockouts
% The preceding line is only needed to identify funding in the first footnote. If that is unneeded, please comment it out.
\usepackage{cite}
\usepackage{amsmath,amssymb,amsfonts}
\usepackage{algorithmic}
\usepackage{graphicx}
\usepackage{textcomp}
\usepackage{xcolor}
\usepackage{hyperref}
\usepackage{ngerman}
\def\BibTeX{{\rm B\kern-.05em{\sc i\kern-.025em b}\kern-.08em
    T\kern-.1667em\lower.7ex\hbox{E}\kern-.125emX}}
\begin{document}

\title{Konzeptpapier\\Bomberman?}

% Authoren	
\author{
	\IEEEauthorblockN{Bösl, Florian}
	\IEEEauthorblockA{
		\textit{f.boesl@oth-aw.de}\\
	}
	\and

	\IEEEauthorblockN{Chernysheva Anastasia}
	\IEEEauthorblockA{
		\textit{a.chernysheva@oth-aw.de}\\
	}
	\and

	\IEEEauthorblockN{Kohl Helge}
	\IEEEauthorblockA{
		\textit{h.kohl@oth-aw.de}\\
	}
	\and

	\IEEEauthorblockN{Korinth Patrice}
	\IEEEauthorblockA{
		\textit{p.korinth@oth-aw.de}\\
	}
	\and

	\IEEEauthorblockN{Porsch Philipp}
	\IEEEauthorblockA{
		\textit{p.porsch@oth-aw.de}\\
	}
}

\maketitle

\begin{abstract}
	Ziel des Papers ist es ein Spiel zu entwickeln bei dem mehrere Spieler in Echtzeit in ihrem Browser gegeneinander antreten können. Das Spielprinzip soll dem Spiel \glqq Bomberman\grqq ähneln.
\end{abstract}

\section{Einführung}



Die Filmbranche ist eine der größten Unterhaltungsindustrien und produzieren eine Vielzahl von Filmmaterial.
Dadurch ist es schwer einen Überblick über alle verfügbaren Filme zu behalten.
Somit werden Filme nur über aktuelle Werbung oder in Gesprächen mit anderen Menschen bekannt.
Um diese beschränkten Möglichkeiten der Filmsuche zu erweitern gibt es Online-Filmdatenbanken und Filmempfehlungssysteme.
Das Projekt ist Teil der Vorlesung "Semantic Web Technologien" an der OTH Amberg-Weiden und soll Filmdatenbanken einfach in mehreren Richtungen durchsuchbar machen.
Dazu sollen, ausgehend von Filmen, Rückschlüsse auf Mitwirkende und umgekehrt gezogen werden können, um Informationen über diese zu erhalten.
Weiterhin sollen die Datenbanken mit Filtern durchsuchbar sein, um Filmvorschläge finden zu können.





\section{Verwandte Arbeiten}

Nachdem in der Einleitung bereits die Motivation erläutert wurde, werden hier verwandte Arbeiten und Systeme vorgestellt.
Es gibt bereits eine große Anzahl an Systemen, die einen gewünschten Film vorschlagen können. Diese Systeme kommen jedoch alle mit gewissen Einschränkungen.
Bei \cite{cinemate} werden Filme aufgrund einer eingegebenen Liste von Filmen vorgeschlagen.
Mit einem Fragenkatalog über Anlass und Stimmung wird bei \cite{pickamovieforme} ein kleiner Datensatz nach einem Vorschlag durchsucht.
Für Informationen und ähnliche Filme in der gleichen Kategorie kann in der Anwendung von \cite{bestsimilar} gesucht werden.
Aktuelle Filme werden bei \cite{tastedive} vorgeschlagen.
Einen genaueren Ansatz versucht \cite{MovieGEN} umzusetzen, bei dem die Vorlieben der Nutzer über Machine Learning und Clusteranalyse verglichen werden, um eine Filmauswahl zu treffen.

\section{Anforderungen}

\subsection{Anforderung 1}
\label{1}
Als Spieler möchte ich ein Spiel eröffnen können, um ... .


\subsection{Anforderung 2}
\label{2}
Als Spieler möchte ich einem Spiel beitreten können, um ... .

\subsection{Anforderung 3}

Als Spieler möchte ich mich auf dem Spielfeld bewegen können, um ... .

\subsection{Anforderung 4}

Als Spieler möchte ich Bomben legen können, um ... .


\subsection{Anforderung 5}
Als Spieler möchte ich das die Bomben explodieren, um ... .

\subsection{Anforderung 6}

Als Spieler? möchte ich das ich bei Kontakt mit anderen Spielern kollidiere, um ... .

\subsection{Anforderung 7}

Als Spieler möchte ich unabhängig von anderen jederzeit spielen können, um ... .

\subsection{Anforderung 8}
Als Spieler möchte ich mit anderen spielen können, um ... .

\subsection{Anforderung 9}
Als Spieler möchte ich mit Hindernissen kollidieren können, um ... .

\subsection{Anforderung 10}
Als Anbieter möchte ich, dass Spiele bei zu langer Spieldauer beendet werden, um ... .

\subsection{Anforderung 11}
Als Spieler möchte ich Power-Ups einsammeln können, um ... .

\subsection{Anforderung 12}
Als Spieler möchte ich ein Leaderboard einsehen können, um ... .

\subsection{Anforderung 13}
Als Spieler möchte ich mit mehr als einem weiteren Spieler spielen können, um ... .

\subsection{Anforderung 14}
Als Spieler möchte ich eine dynamische Spielfeldgröße, um ... .

\subsection{Anforderung 15}
Als Spieler möchte ich meinen Spielernamen festlegen können, um ... .

\subsection{Anforderung 16}
Als Spieler möchte ich meine Spielfigur personalisieren können, um ... .

\section{Methoden}

\subsection*{Über die Daten}

In einer lokalen Graphdatenbank werden die aktuellen Filme und Serien diverser Streaminganbieter (Netflix, Amazon Prime, Hulu, Disney+) abgespeichert. Außerdem werden diese durch Informationen aus der IMDB (Internet Movie Database) ergänzt. Mithilfe dieses Datenbestandes soll es möglich sein, komplexe Abfragen zu formulieren. Durch den SPARQL-Endpoint der DBpedia ist der Zugriff auf weitere Details möglich.

\subsection*{Backend}

Als Backend wird ASP.NET Core verwendet. Dieses von Microsoft entwickelte Framework hat sich in der Welt der Microservices in den letzten Jahren etabliert. Über noch nicht definierte Graph Query Languages soll auf die verwendete Graphdatenbank bzw. DBpedia zugegriffen werden können, um Informationen zu den Filmen und Schauspielern auszulesen.

\subsection*{Frontend}

Für die Interaktion mit dem Benutzer wird React verwendet. Über eine REST-Schnittstelle des Backends kann auf benötigte Informationen zugegriffen werden.

%\section*{Referenzen}

\begin{thebibliography}{0}
	\bibitem{cinemate}Cinemate [Online] \url{https://cinemate.me/} (visited on Nov. 15, 2021)
	\bibitem{pickamovieforme}Pickamovieforme [Online] \url{https://pickamovieforme.com/} (visited on Nov. 15, 2021)
	\bibitem{bestsimilar}Bestsimilar [Online] \url{https://bestsimilar.com/} (visited on Nov. 15, 2021)
	\bibitem{tastedive}Tastedive [Online] \url{https://tastedive.com/movies} (visited on Nov. 15, 2021)
	\bibitem{MovieGEN}MovieGEN [Online] \url{http://citeseerx.ist.psu.edu/viewdoc/download?doi=10.1.1.703.4954\&rep=rep1\&type=pdf} (visited on Nov. 15, 2021)
\end{thebibliography}
\vspace{12pt}

\end{document}
